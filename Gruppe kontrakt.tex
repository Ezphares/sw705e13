\chapter{Gruppe kontrakt}
\label{sc:grpKontrakt}

\textbf{Medlemmer:}
\begin{itemize}
	\item Christian Klim Hansen - Dataintensitive Systemer & Agenttech
	\item Anders Vinther   - ? & Agenttech (Kontakt)
	\item Barbara A. Flint - Webengineering & Agenttech (Slavepisker/Referant)
	\item Jeppe Torp       - Webengineering & Agenttech 
	\item Simon Jensen     - Webengieering & Agenttech
\end{itemize}

\section{Forventninger til gruppemedlemmer}
\label{sc:forventninger}

\subsection{Arbejdsregler}

St� ved dine pauser
Frokost fra 12:00-12:30
Det forventes at folk dokumentere/forklare deres kode, f�r det bliver committet.

\subsection{M�detid}

Onsdag (gruppem�de og vejlederm�de)
Klokken 9, medmindre grupperummet er reserveret til gruppearbejde.
Folk bestemmer selv, om de kommer til forl�sninger og opgaveregning.
Folk giver besked (SMS), hvis de ikke kommer til opgaveregning. (besked senest kl. 21:00 dagen f�r eller en time f�r opgaveregning starter)

\subsection{Agenda for m�der}

Fastl�gges p� dagen

Vi �nsker f�lgende til hvert m�de:

\begin{itemize}
	\item Status (hvor langt er vi, hvad har vi lavet siden sidst)
	\item Nye opgaver (hvad skal vi n� til n�ste m�de)
	\item Varmstol er p� pr�veordning
\end{itemize}

Agenda til vejlederm�de udarbejdes og sendes i fornem tid!

\section{Versionsstyring}

Vi anvender Git, som Jeppe s�tter op.

\section{Generelle ting}

Husk at g�re rent i rummet efter sig selv.
Fredag er "hyggedag" starter omkring kl. 14-14:30.

\section{Kommentarer}
Anders foretr�kker ikke nazisme
Send besked, hvis man ikke kommer til opgaverregning
God ide at give lyd fra sig
Bedre kommunikation imellem medlemmerne


