\section{Game Rules}
\label{sec:game_rules}

This section describes the different concepts of the game, and how they interact, as well as the semantics of the various instructions possible.

\subsection{Elements}

\subsubsection{Board}
The game board is a hexagonal grid inside the game world. This grid has a \emph{size}, which describes how many \emph{tiles} are along each side of the board.

\subsubsection{Position}
A position is a hexagonal segment of the game world. Each position is adjacent to 6 other positions. A position can be either inside a game board, in which case it is called a \emph{tile} or outside, which is referred to as an out of bounds, or \emph{OOB} position.

\subsubsection{Direction}
Directions describe the relation of adjacent positions. In the hexagonal grid, 6 directions exist: Left, Right, Up-Left, Up-Right, Down-Left, Down-Right.

\subsubsection{Tile}
A tile is a \emph{position} inside the game board. A tile can be either empty, or occupied by a single \emph{entity}. If two entities are inside the same position, they will \emph{battle}.

\subsubsection{OOB}
Out of bounds is the name for any positions outside the game board. Unlike \emph{tiles}, OOB positions have no concept of empty or occupied. Any entity trying to occupy an OOB position ceases to exist.

\subsubsection{Entity}
An entity is an actor in the game world. An entity can be either a \emph{cell} or \emph{food}. No matter the type of entity, they have the following properties:
\begin{itemize}
\item A \emph{position}.
\item An energy count. If this becomes zero, the entity ceases to exist.
\end{itemize}
Each entity gets a \emph{turn} in order. The turns progress in order of oldest to youngest in the game world, and then resets to the oldest again. Each entity loses one energy at the end of its turn.

\subsubsection{Food}
Food is the simplest \emph{entity}. It can take no action on its turn, and it always loses battles.

\subsubsection{Cell}
Cells are advanced \emph{entities}. In addition to the properties common to all entities, they have several more:
\begin{itemize}
\item A \emph{program}, determining its actions.
\item A team, determining which other cells it considers friend, and which it considers enemies.
\item A set of \emph{variables}, three of each type.
\item The instruction pointer - Information about which \emph{instruction} of its program to start \emph{executing} from.
\end{itemize}
Apart from the above a cell can take an action during its turn. It first \emph{executes} its program, then either does nothing, \emph{moves} or \emph{splits}.

\subsubsection{Program}
Like a board, a program is a hexagonal grid. Instead of tiles, the program consist of \emph{instructions}. A program can be \emph{executed}, starting from a given instruction. A program is created by a player before the game starts.

\subsubsection{Instruction}
One or more instructions make up a \emph{program}. Each instruction has an \emph{instruction type} as well as several parameters pertaining to that type. When an instruction is up for execution by a program, different things may happen, according to instruction type.
\begin{itemize}
\item Variable changes - an instruction may update the values of a variable.
\item Instruction pointer update - the instruction will tell the cell which instruction to execute next. All instructions define a direction to \emph{continue}, while instructions representing control structures also define a direction to \emph{divert}. The chosen direction will describe to the program how to update the instruction pointer.
\item Action Choice - If an instruction results in an action, the cell will take that action for the turn, and program execution will end until the next turn of that cell.
\end{itemize}
Each instruction is described in \autoref{sub:instructions}.

\subsubsection{Variables}
Each entity has 9 variables that can be changed and read by the program. These variables are split into 3 types. Each type and its domain is as follows:
\begin{itemize}
\item \emph{direction} - domain: \{ 'left' , 'right', 'up-left', 'up-right', 'down-left', 'down-right' \}
\item \emph{entitytype} - domain: \{ 'friend' , 'enemy', 'food', 'empty-tile', 'out-of-bounds' \}
\item \emph{number} - domain: $\mathbb{Z}$
\end{itemize}
Whenever an instruction needs a value of either of these types, it is possible to have it set explicitly, have it use the value stored in one of the variables, or use a derived value of the given type.

\subsubsection{Derived Values}
Derived values are values that change depending on the game state. An example of a derived value of type \emph{number} would be the cell's current energy, while an example of type \emph{direction} would be the last direction the cell moved.

\subsection{Actions}

\subsubsection{No Action}
The entity does nothing. This happens when an executing program does not reach an instruction resulting in an Action Choice.

\subsubsection{Move}
The entity moves to an adjacent position, described by a \emph{direction}. If this position is occupied the moving entity will \emph{battle} the occupying entity.

\subsubsection{Split}
The entity splits into two entities. The new entity will spawn at an adjacent position, described by a \emph{direction}. Each entity will have half the energy of the original entity. In case of an uneven amount of energy, the new entity will receive the least amount. The new entity will have the same program as its parent, but will start the program from the beginning. All variables of the new entity will be set to default values. If the new entity spawned on an occupied position, it will immediately battle the occupying entity. The new entity will not get to take any action the turn it was spawned.

\subsection{Events}

\subsubsection{Battle}

Battle takes place whenever an entity (the attacker) moves to a position that is already occupied by another entity (the defender). As a position can be occupied by only one entity at a time, one of the entities must be removed. Battles happen even if the two entities are cells from the same team. The decision is made as follows:
\begin{itemize}
\item If the defender is \emph{food}, the attacker wins.
\item If the defender has less than or equal energy to the attacker, the attacker wins.
\item Else, the defender wins.
\end{itemize}
When a winner has been chosen, it receives all of the loser's energy, and the loser ceases to exist.

\subsubsection{Program Execution}

Each cell executes its program once per turn. When the program executes, it will look up an instruction depending on its current state, execute the instruction, and select the next instruction according to the \emph{divert} or \emph{continue} information of the executed instruction. This will continue until one of two things happen:
\begin{itemize}
\item It reaches an instruction resulting in an \emph{Action Choice}, in which case it tells the executing cell to perform that action.
\item It has executed an amount of instructions this turn equal to the instruction limit. In this case execution is suspended for a turn and the cell is told to do nothing. This is done so that infinite loops without actions do not make the game impossible to complete.
\end{itemize}

\subsection{Instructions}
\label{sub:instructions}

\subsubsection{No Operation}
The no operation instruction causes no variable changes or action choices, and defines only a \emph{continue} direction.

\subsubsection{Move}
The move instruction causes an action choice. The action will be \emph{move} in a \emph{direction value}. Move defines only a \emph{continue} direction.

\subsubsection{Split}
The split instruction causes an action choice. The action will be \emph{split} in a \emph{direction value}. Split defines only a \emph{continue} direction.

\subsubsection{Look}
The look instruction causes variable changes. It will look at an adjacent position given by a \emph{direction value} relative to the executing cell. It will store the entity type found in a chosen \emph{entitytype} variable, and the amount of energy found in a chosen \emph{number} variable. Look defines only a \emph{continue} direction.

\subsubsection{If}
The if instruction causes no variable changes or action choices. It defines a type, either \emph{entitytype}, \emph{direction} or \emph{number} as well as a boolean expression for two \emph{values} of the chosen type. Apart from the values, the expression has an operator, which for \emph{entitytype} and \emph{direction} can be either \emph{equal} or \emph{not equal}. For  \emph{number}, additional operators are available: \emph{less than}, \emph{less than or equal}, \emph{greater than} and \emph{greater than or equal}.
If defines a \emph{continue} direction for use when the expression evaluates to true, and a \emph{divert} direction for times when the expression is false.

\subsubsection{Loop}
The loop instruction causes variable changes. It defines a type, either \emph{entitytype}, \emph{direction} or \emph{number}, as well as two \emph{values} of the chosen type, called \emph{start} and \emph{end}. Additionally, it defines a loop direction depending on type. \emph{number} can be either +1 or -1. \emph{direction} can be either clockwise or counter-clockwise. \emph{entitytype} can only be a special value, \emph{nextentity}, which simply chooses the next entity type in a specified order. It also specifies a \emph{variable} of the chosen type, called the iterator. When a loop is executed, it can either be started, continued or completed, depending on the following conditions:
\begin{itemize}
\item If the loop instruction is executed for the first time in a turn, it is started.
\item If the loop is executed for the first time since it was completed, it is started.
\item If neither of the above is true and the loop is executed while the iterator is set to the end value, it is completed.
\item If it is neither started or completed, it is continued.
\end{itemize}
Loop defines both a \emph{continue} direction and a \emph{divert} direction, and acts as follows:
\begin{itemize}
\item If the loop is started, the value of the iterator is set to the start value, and the \emph{continue} direction is chosen.
\item If the loop is continued, the value of the iterator is modified according to the loop direction, and the \emph{continue} direction is chosen.
\item If the loop is completed, no variable changes are made, and the \emph{divert} direction is chosen.
\end{itemize}
