\section{Programming the Cell}\todo{Expand}

Programming the cell is an essential part of the game. This is where the user will learn basic construct in the imperative paradigm by using them to program the initial cell and all future copies of that cell.\newline

The game begins with the player programming the initial cell in a drag-and-drop block interface similar to that of Carnage Heart.
The player can add, declare variables, for-loops, conditional statements, and standard action functions with parameters, see \autoref{sec:designingBehaviorCells}.
Programming the cell should not be the cause of frustration due to the occurrence of peculiar programming-oriented errors, which the player will have to deal with.
For this reason, it is important that the interface provides the user with feedback which gives a clear understanding of when the user has combined a string of legal or illegal programming constructs.
This can be done by implementing a basic green is correct and red is wrong color schema, that allow the user to immediately identify good code and code that includes one or more errors.
For this to work, the colors will have to be dynamic and evaluation of constructs in the program must be done on a regular basis to keep providing the player with feedback about whether the program is correct or wrong.
If the colors are not update regularly, a construct might be flagged as red/wrong, the user corrects it, so that the construct is valid, but the construct is still flagged as red/wrong.
In general terms, it is important that the user cannot make too many construct mistakes.
Otherwise, the game is in danger of becoming a debugging game, when it should be a game about creating the most optimized and best algorithm/cell.