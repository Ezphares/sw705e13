\section{Game Environment}
\label{Sec: Game Environment}

Given the nature of the game it is important that it seems immediately accessible to the user, in the sense that they get a good feeling of how the game 
is supposed to work, and how they can achieve success within the game. The challenge will then be to create a game  which manages to educate, while is 
in its own right rather exciting to the user. The following section will describe the initial ideas had about the game, and how a solution that 
achieves this could look like in the early stages of design.

\subsection{The Gameplay}

The game is intended to be a competitive 'who-has-the-best-algorithm' to solve a problem on a playing field type of game. The player will be in charge 
of a simple biological cell, which can consume food on the playing field, and evolve and grow larger. The game finishes when one player has grown 
larger than the opponent's and is able to eat the other player. There will be multiple options available to each player, for example; they may chose to 
split-up their cell - evenly partioning their cell and creating a copy, allowing the player to control more cells on the playing field, they could move 
it in a given direction, they could try to attack the other player, or perhaps they go scavenging for food. The player programs their cell(s) before a 
game begins in an easy-to-use drag and drop interface.

\subsection{Programming the Cell}

Programming the cell should be inviting for the user, and not cause frustration due to the occurance of peculiar programming-oriented errors, they will 
be tasked to dealing with. For this reason it is important that the user interface gives the player a clear understanding of when they have combined a 
string of illegal programming constructs but in turn also shows when they have combined a string of legal constructs. In general terms however it is 
important that the user cannot make too many mistakes, and instead of becoming a game about debugging, it becomes a game about creating the most 
optimized and best algorithm.

\subsection{The Grid}

The Grid is the map on which the cells compete, and is split up into tiles. Splitting the map into tiles will make it easier to solve, whether a cell 
is wihin range of another cell, or whether a given cell is standing on food on the grid or some other item of interest from a developer standpoint. 
Doing this will also take a toll on the freedom the player has in terms of moves and gameplay however, in the sense they cannot move a complete 360 
degrees or make exactly the decision they may want to make. So as a compromise to this, the group decided to use hexagonal tiles as seen in the 
blockbuster game title Civilization 5.\todo{Find eventuelt den talk Sid Meier afholder omkring Hexagons i CIV 5 og hvorfor det er superr duperr}