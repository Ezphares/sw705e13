\section{Client-Server Model}
Our application will make use of the Client-Server model, a model that consists of two parts, a client and a server.

The client runs on the users PC, and will be rendering graphics and gameplay in the users browser. This is done locally, without the need to connect to a server. The client will connect to the server over the internet to receive information such as leaderboard statistics. Multiple clients could be able to play against eachother by connecting to a central server, the server would then match them against eachother in the same game and keep track of their progress. Depending on how it is implemented, the server could also be running the clients programs due to multiple reasons. A big reason as to why we would want the server to be running players programs is to prevent cheating. If the server executes the programs, situations where the client is modified to increase the number of points given or the time taken to solve a problem could be avoided.

Instead of using the Client-Server model, the Peer-to-peer model could have been chosen. The Peer-to-peer model is like the Client-Server model without the server, instead all the clients connect to eachother in a decentralized system. This is a good system because it drastically reduces the cost that a Client-Server based system has to keep servers running. The Peer-to-peer model also has some negatives, one of these negatives are that because most of the peers in a Peer-to-peer system are PC's, they may not always be available, and thus if no or few people are playing the game it becomes difficult to play at all. Another negative is that the prevention of cheating becomes more difficult, it is possible to have player clients run eachothers programs, but modified clients can still take over the system and corrupt the results.