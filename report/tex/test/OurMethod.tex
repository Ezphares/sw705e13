\section{Implemented Testing Method}
\label{sec:test_method}

The following section will describe how we set up the testing environment for our users, how we created our test, and how we intend to analyze the data.
Finally we will describe the limitations of our testing experience.

\subsection{The Set-up}

The test will be conducted by requesting a server from the University to host our game, which will then be available to the outside world by visiting the website \textbf{www.simonjensen.net}.

Making the game available to the entire world can enable us to expose the game to a larger audience, than the alternative, which could be to schedule a formal Usability Laboratory evaluation.

\subsection{Testing the System - Game System Experience}

We have decided to utilize \textbf{Unit Tests} to check that the code works as intended.
The more code we are able to unit test, the more we ensure a high code coverage on our tests.

\subsection{Testing the Player - Individual Player Experience}

Given that the test is going to be available to the entire world, we can potentially open up feedback for many more people, than we are able to interview directly. Therefore we have chosen to use questionnaires and written feedback which we request from the users when they have played our game, by following a link to a survey. Furthermore, given that the games test is advertised on various Internet forums, user comments from these will also be taken into consideration for evaluation.

\subsection{Limitations of the Testing Method}

When we take into account the GX method, it is obvious, that there are things we cannot test for, since it is both out of scope for this project and the university may not be able to provide the needed technology.
For example, we do not include testing of player context.
It would only be viable for us to test, how well the game plays while on-the-move, in terms of whether the lights and colors on the screen are clear enough. Furthermore, the game in its current state does not work on a mobile platform.
Hiring a cultural expert to aid with cultural debugging is out of scope.


In terms of evaluating the player, it is not possible in this project to set up a psycophysiological test, as that would demand access to hardware that we do not currently possess. We have chosen an open testing set-up, which means that we will not have direct access to conversations with our testers, and understand in greater depth their issues with the program. Rather, it is possible to communicate with them, but we cannot be certain they will respond. Furthermore, we cannot guarantee that every tester, actually fills out the questionnaire. However, we get a much more informal testing method for our players, and we hope the availability will compensate and uncover the more obvious problems with our software.

\subsection{Questionnaire}

The questionnaire is based around the Likert scale, letting people choose between various levels of agreement for each question. This approach is faster for a tester to fill out, compared to writing each answer in detail and gives more freedom in how they choose to reply. Of course, letting the players entirely describe each answer is an option as well, but this may be too time consuming for users, when they are asked to test in this informal and casual way. The questionnaire can be seen in Appendix \ref{app:questionnaire}.