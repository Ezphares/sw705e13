\section{Edutainment for Private Consumers}
\label{sec:privateconsumers}
Many companies create educational games for children to learn mathematics, languages and other basic school topics.
Most of these games are targeted towards younger children, who are either about to start school or in elementary school.
A company like Krea Games creates multiple series for children of various ages.\cite{kreagames}
These games are mostly available for PC, with a small subset also being available for Android and iOS devices.
In this section, we will examine some of the best selling educational games for private consumers.

\subsection{Best Selling Edutainment Games}
One of the most popular edutainment games is 'The Oregon Trail' with more than 65 million copies sold worldwide since its initial release in 1982.\cite{oregontrail}
The game was originally released to teach children in American schools about how pioneers lived in the $19^{th}$ century, but was also released to the general public as a standalone game for the PC. 'The Oregon Trail' was one of the first video games ever used in the United States for educational purposes, which has added significantly to its popularity.

The objective of the game is to help a group of American pioneers follow the Oregon trail and make sure, that they survive their endeavor to finally settle down in Oregon. The game teaches children the history of the Oregon Trail without sacrificing the gameplay experience and thus the game is considered a good game even without the educational aspect.\newline

As with 'Global Conflict', the game includes a sense of realism.
Random events play a large role in Oregon Trail.
It is possible to die from various diseases (such as dysentery), cattle being stolen, food rations getting low etc.
It seems, that focusing on making the game feel more realistic to the player is part of the engaging nature of both games.
Neither of the games impress in terms of graphics however both games do possess a sense of realism. From this it can be extrapolated that it is less important how the game looks graphically and more important to focus on creating a realistic situation, that the players want to interact with and learn more about.\newline

Other popular edutainment games include:
\begin{itemize}
\item "Where in the World Is Carmen Sandiego?", which is a game released in 1985, that teaches geography.
The game was popular enough to spawn several sequels released as late as in 1998 and the first version of the game was released in 2011 as a Facebook game.\cite{carmensandiego}
The game is centered around the detective Carmen, who travels around the world in order to capture criminals.
Learning geography is an essential part of the game, since the player has to follow clues to where in the world a thief, she needs to catch, has gone. Later games in the series taught different subjects such as history or mathematics. In order to become a better player the user has to improve their knowledge of the topics in the game, which is extremely useful and shows that the process of learning while having fun is definitely possible and attainable in video games.

\item "Lemonade Stand" is a game created by Bob Jamison and released in 1973, which teaches the player about economics.\cite{lemonadestand}
It is a game, where the players play as a child who has a lemonade stand.
The player makes choices regarding the cost of the lemonade, money spent on advertising, and how many glasses of lemonade to make, thereby teaching basic economics in a fun way, that children in some cultures can relate to.
The everyday situation of the game makes it very relatable. 
Even though lemonade stands are not a common sight in Denmark, the aspect of making the game, so that the user can relate to the situation, seems to be an important factor.
Maybe this is one of the aspects that have made the game popular enough that several remakes of the game has been released throughout the years.
\end{itemize}

By analyzing and comparing these games, it is possible to create a game that is both engaging and educational, while avoiding making it too obvious that the game is trying to teach the player something.
Thus the player does not feel like they are doing homework while playing the game.
We have seen, that the aspects of realism in 'Global Conflicts' and making the game relatable in 'Lemonade Stand' may be part of the reason for the success of these games, and that, if possible, these aspects should be included in our game.

\subsection{Browser-based Edutainment Games}
Most edutainment games, that can be accessed from a browser on the PC are flash games on websites.
Some of these claim to be approved by teachers for educational purposes.
Many of the games that we have found are heavily inspired by classic games, such as a Pac-Man like game called Math Man, where the player has to play Pac-Man while doing math problems to win.\cite{mathman}
Grand Prix Multiplication is another example, which is a multiplayer racing game, where the player has to do multiplication to make the car drive faster in order to win.\cite{grandprix}\newline

Browser-based games are good in terms of availability, since most households have a PC and in many cases, each family members has their own device with access to the Internet. Most browser-based games are free-to-play, which increases their availability. However, our overall impression of the browser-based edutainment games is, that they are very explicit in their teaching method. This can be a good thing if the game is primarily used in schools, where the purpose is to teach new material. However, the explicit teaching nature of these games can make them seem like homework and keep the children from playing the games in their spare time. We found, that in general the games we look at neglected the entertainment and engaging aspect of video games. It is important to avoid explicit teaching in our final product.