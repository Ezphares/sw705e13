\section{Educational Games for Private Consumers}
\label{sec:privateconsumers}
Many companies create educational games for kids to learn math, languages and other basic school topics. Most of these games are targeted small kids who are about to start school, or are in elementary school. A company like Krea Games\cite{kreagames} creates multiple series for differently aged children, these games are mostly available for Personal Computers, with a small subset also being available for Android and iOS devices.

One of the most popular educational games is `The Oregon Trail', released first in 1982 with more than 65 million copies sold worldwide since.\cite{oregontrail} The objective in the game is to help a family of american pioneers follow the Oregon Trail and help them survive their endeavor and settle down in Oregon.
The game `The Oregon Trail' intends to teach children about the life of the pioneers on the Oregon Trail in the middle of the 19th century. The game manages to teach children the history of the Oregon Trail without sacrificing the gameplay experience, and thus the game is considered a good game even without the educational aspect.

Most educational games that can be accessed from a browser on the family computer are flash games on websites, some of which claim to be teacher approved. These games are usually pretty bad versions of classic games, such as a Pac-Man like game called Math Man where the player has to play Pac-Man while doing some math problems to win.\cite{mathman} Another example of a browser-based educational game is Grand Prix Multiplication, a multiplayer racing game where the player has to do multiplication to make the car go faster and win.\cite{grandprix}

The overall impression of the browser-based educational games is that they are very explicit in their teaching method. This can be a good thing, for example it could be useful in school to actually teach children new subjects, but it could easily seem like homework and keep the kids from playing the games when they are at home.

The prices of educational games varies from platform to platform. Browser-based games are usually free, games on tablets ranges from free to about 30 DKK. Downloadable PC games cost from 30 DKK to around 100 DKK. Prices on educational games are as such significantly lower than prices of regular games.