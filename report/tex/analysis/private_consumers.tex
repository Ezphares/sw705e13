\section{Educational Games for Private Consumers}
\label{sec:privateconsumers}
Many companies create educational games for children to learn math, languages and other basic school topics. Most of these games are targeted towards younger children, who are about to start school or are in elementary school. A company like Krea Games \cite{kreagames} creates multiple series for differently aged children. These games are mostly available for PC, with a small subset also being available for Android and iOS devices.

\subsection{Best Selling Educational Games}
One of the most popular educational games is ``The Oregon Trail'', released first in 1982 with more than 65 million copies sold worldwide since.\cite{oregontrail} The objective in the game is to help a family of American pioneers follow the Oregon Trail and help them survive their endeavor and settle down in Oregon.
The game `The Oregon Trail' intends to teach children about the life of the pioneers on the Oregon Trail in the middle of the 19th century. The game manages to teach children the history of the Oregon Trail without sacrificing the gameplay experience, and thus the game is considered a good game even without the educational aspect.

Other popular educational games are ``Where in the World Is Carmen Sandiego?'' from 1985 with many sequels released as late as 1998.\cite{carmensandiego} A facebook version of the first game was released on Facebook in 2011. And ``Lemonade Stand'' created by Bob Jamison in 1973.\cite{lemonadestand}
The game ``Where in the World Is Carmen Sandiego?'' teaches geography by making the player follow clues to where in the world a thief they need to catch has gone. Later games in the series teach different subjects such as history or math.
``Lemonade Stand'' is a game where you play a child that has a lemonade stand. The player has to make choices regarding the cost of the lemonade, money spent on advertising and how many glasses of lemonade to make. The game teaches economics without feeling like schoolwork, instead the player needs income to continue playing the game. Many remakes of this game has been released throughout the years, they are still very popular.

\subsection{Browserbased Educational Games}
Most educational games that can be accessed from a browser on the family computer are flash games on websites, some of which claim to be teacher approved. These games are often heavily inspired by classic games, such as a Pac-Man like game called Math Man where the player has to play Pac-Man while doing some math problems to win.\cite{mathman} Another example of a browser-based educational game is Grand Prix Multiplication, a multiplayer racing game where the player has to do multiplication to make the car go faster and win.\cite{grandprix}

Our overall impression of the browser-based educational games is, that they are very explicit in their teaching method. This can be a good thing, for example it could be useful in schools to teach children new subjects, but it could easily seem like homework and keep the children from playing the games when they are at home.

The prices of educational games varies from platform to platform. Browser-based games are usually free, games on tablets ranges from free to about DKK 30, where both platforms are possible subjects to a form of micro transaction system. Downloadable PC games cost from DKK 30 to around DKK 100. Prices on educational games are as such significantly lower than prices of regular games which usually cost around DKK 400.\todo{Maybe missing source}