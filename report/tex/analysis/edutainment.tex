\section{Edutainment}

Edutainment, a portmanteau of the words "educational entertainment" is term used for entertainment made with the purpose of teaching the user something.

Edutainment is still debated as a term. Mitchel Resnick of MIT Media Laboratory dislikes it because we teach children from a young age that education is a passive activity, a service provided to you. And one that is boring and should be livened up with entertainment - another word he claims has passive, service-oriented connotations. Instead he prefers the term "playful learning", because playing and learning are activities that focus on the process instead of being means to an end. The key to successful playful learning, he says, is to use the child's intrinsic motivation and curiosity around a subject to help them learn about it. The challenge, he says, is that many people today view games and play as not having any value when it comes to teaching people anything. 

In Mitchel Resnick's definition, what this project should achieve is playful learning, however it will continue to be called edutainment, due to the fact that his definition of playful learning is the same as the group's definition of edutainment. The properties of edutainment should be that it is

\begin{itemize}
\item Intrinsically motivated, i.e. the child should want to do the activity, not have an external source of motivation for it such as a teacher saying they must do it.
\item Focus on the creative process of trial and error without necessarily having any scientific approach or pre-existing hypothesis to test.
\item Give the child hands-on experience with the subject matter (e.g. mathematics or physics).
\end{itemize}\cite{edunoty}

While the target audience in this project  may not be children, the properties listed above can also apply to older people, as curiosity and creativity are traits that do not disappear with age.
