\section{Problem Statement 1}
\label{sec:problemstatement1}

Routines are a big part of many peoples lives and it is in many ways a natural way of approaching problems. Recipes are used for making food, instructions followed to assemble a pieces of furniture and daily routines are created to make repeated tasks easier. This approach to problem solving is also common in software engineering, however people do not realize the use of this algorithmic way of thinking. Few people know how programming works, and that the way they often create routines in their daily lives can be translated into small programs with a specific goal.

Edutainment games is a genre of games, that has existed for some time, but it is only within the last 10 years, that it has been actively used as an alternative approach to traditional learning of basic skills, such as language and math. Many games focus on a younger consumer base, around seven years old. A lack of focus on more advanced topics and older users limits the learning potential. This project focus on introducing the advanced topic of programming into a edutainment environment for older children, that is available for the users and teach basic skills in programming while not teaching a specific programming language.

\textbf{Proposal 1}
\begin{quotation}
	How can we design a edutainment game focused on an older consumer base, that teaches people some of the basic elements of programming without making it seem like hard work, but rather a fun way to approach learning? What techniques can be utilized to make the game available to a large amount of potential consumers, and make the consumers want to return to the game for further improvements? Which gamification techniques should be implemented and which should not, to make the game engaging for the user?
\end{quotation}

\textbf{Proposal 2}
\begin{quotation}
	How can gamification theory, edutainment games and programming be combined, to make a game that teach older children basic skills in programming and give them the ability to solve problems with a natural algorithmic approach? What limitations on availability should be considered and how does increasing availability effect the potential quality of a game? How can the amount of new and familiar elements be balanced to make the game fun to play?
\end{quotation}

