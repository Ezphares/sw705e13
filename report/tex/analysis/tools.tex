\section{Development Tools}
\label{sec:tools}

When developing a browser based game, several different technologies are available.
Six of these are of some interest; HTML5, Flash, Java, Silverlight and Unity, since they provide easy development of internet graphics, games and animations.
They are also some of the most well known development tools in the development of internet applications. The reason that these specific technologies are so widely used, is that that each have had a large focus on applications embedded in browsers, either natively, as is the case of HTML5, or through plugins.
The development tools for Flash are, however, not free \cite{adobe13}, and as such not an option for this project.
Besides, Flash has been discontinued for mobile platforms \cite{adobe12}, and would limit the availability of the game, which, as mentioned in \autoref{sec:availability}, is an important consideration.
The rest of the tools will be introduced in this section, mostly focusing on the portability of each tool, as measuring differences in performance and prevalence of required plug-ins and the like is virtually impossible.
Another focus, given the scope of the project, is that the game should exist solely as a web application, and not a standalone game, app or similar.

\subsection{HTML5, Canvas and WebGL}
HTML5 is the latest standard for implementing web-pages, and the current recommendation from \ac{w3c}, \cite{html513}.
It includes several useful APIs for implementing multimedia, specifically the canvas element, which provides an area, where graphics can be drawn.
HTML5 games will be written in javascript, which is interpreted by the browser, and as such is available on most computer systems.\newline

The canvas can be further improved using the WebGL extension,\cite{khronos13}, which allows for hardware accelerated graphics in the canvas. 
This will eliminate much of the low performance, that can arise from software accelerated graphics.
Additionally, WebGL is implemented by default in all modern browsers, including some mobile browsers, so most users will be able to play the game without downloading any additional software.

The main disadvantage of a HTML5-based solution is, that it is a relatively new technology, and for that reason problems could occur, where an optimal solution has yet to be found, which could increase development time. Various browsers support HTML5 differently. The website \url{html5test.com} tests the used browser for html5 support and provide a score from 0 to 500, where a score of 500 means, that the browser support all aspects of html5. What we are most interested in is the support for canvas and WebGL. The following table shows the result of various popular browsers.\todo{Could you be a bit more specific which browsers do not support HTML5? Maybe we could visit html5test.com on various devices and compare result?}

\todo{Complete tabular with results from various browsers}
\begin{center}
\begin{tabular}{|l|l|}
\hline
\multicolumn{2}{|c|}{HTML5 Support}\\
\hline
Feature & Supported by \\ \hline

\multicolumn{2}{|c|}{\textbf{Canvas}} \\ \hline
Canvas element &  \\
2D context     &  \\
Text           &  \\ \hline
\multicolumn{2}{|c|}{\textbf{WebGL}}  \\ \hline
3D context & \\
Native binary data & \\\hline
\end{tabular}
\end{center}

\subsection{Java}
With Java it is possible to create applets that run in a browser.\cite{java13} These applets can be used for any type of applications, including games.
The greatest advantages of using Java for applets is its large user base and library collection, which means that a lot of the technical difficulties in building a game has been solved by other people, and open source solutions are publicly available.

The disadvantages of Java is, that it is an external browser extension, and one which is not installed as a standard on most systems.
This means that the user will have to install and keep the extension updated themselves to use the game. 
This could also be an annoyance for institutions having to keep several computers up to date, especially since Java is updated quite frequently, often close to $10$ times each year.\cite{javahistory13}
Additionally, to get it to work on Android systems, it needs to be in the app store, and for iOS systems, Java support is limited in general.

\subsection{Silverlight}
Silverlight is very similar to Java applets, but developed by Microsoft and typically programmed in either C\# or Visual Basic.\cite{silverlight13}
Though Silverlight has historically had a focus on media streaming, today it is a multipurpose framework and most of the same advantages and disadvantages as Java had also apply to Silverlight.
A choice between Silverlight and Java is therefore mostly based on personal preference on the side of the developers.

\subsection{Unity}
Unity is a game development framework, capable of publishing games to browsers.\cite{unity13} 
While Unity is focused mainly on developing 3D games, there is support for any type of game.
Using Unity would provide a useful framework for game development, but distribution could be problematic.
While Unity games can be ported to virtually any OS, desktop or mobile, the browser plug-in is only available for Windows and MacOS.
For the rest of the platforms, the game has to be released as a standalone application, which goes against creating an available web application.

\subsection{Conclusion}

An overview of the compatibility of each technology described, on various platforms can be seen in \autoref{fig:technologies}

\begin{figure}[ht]

\begin{tabular}{|r|l|l|l|l|l|}
\hline
 & Windows & Linux & MacOS & Android & iOS \\
\hline
HTML5 & Full & Full & Full & Partial & Partial \\
\hline
Java & Plugin & Plugin & Plugin & Standalone only & Limited \\
\hline
Silverlight & Plugin & Plugin & Plugin & Plugin & None \\
\hline
Unity & Plugin & Standalone only & Plugin & Standalone only & Standalone only \\
\hline
\end{tabular}
\label{fig:technologies}
\caption{Technology compatibility overview}
\end{figure}

All of these technologies are capable of very similar functionality and portability. This means, that the decision was mainly based on how easy it is to distribute, as well as personal preference of the developers.
In the end, HTML5 and WebGL was the chosen platform.\todo{maybe quickly list of why it has chosen. Or just say that based on the sections above, we evaluate HTML5 and WebGL to be the best option.}