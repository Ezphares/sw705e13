\section{Motivation}
\label{sec:motivation}

Many people interact daily with computers without knowing how a computer works.
There are many aspects of how a computer works, one being how programs are written.
Software is included in an increasing number of products, and it is a field that keeps expanding.\cite{idaArtikelMangel}
It is therefore important, that we get more people to realize, that computers are not too complicated and dangerous to work with.
We might be able to get more people interested in learning programming languages, if we provide a transition tool, that teach people basic programming constructs, and how these constructs can be combined to form working programs.
We define a transition tool as a stepping stone from one level of understanding of a subject to the next.\newline

Games are sometimes used as a learning tool, where the purpose is to have fun while learning.
However, we feel that many edutainment games (entertaining educational games) does not engage students sufficiently into learning and few exist that focus on older children. 
We are therefore motivated to create an engaging edutainment game as the transition tool, where the target consumers are anyone how wants to learn basic programming skills and do not have a higher degree of understanding of math. This is relevant for the paradigm, that we choose to focus on, which is the imperative paradigm. See \autoref{sec:interview} for a discussion of what paradigm to focus on.\newline

With the introduction of tablets and applications, the market for games has increased, and many edutainment games are available on these platforms; iOS, Android, and Windows mobile.
It is important for us, that the product developed during this project can be used on many platforms and are therefore not restricted to for example Apples App Store and Google Play.
The motivation for creating a highly available product is to provide many people with a transition tool in the form of a web application, that makes basic programming skills more common.\newline

It is important to state, that the motivation is not to teach users a specific programming language.
Rather it is important for us, that constructs in mostly imperative languages are understood, such as if-else-statements, for-loops, variable types and handling of variables, and that the user can get a visual representation of when a program is executed. We have chosen to focus on constructs in the imperative paradigm, since it is the first paradigm software engineering and computer science students of Aalborg University are introduced to and based on your interview with Kurt N{\o}rmark, see \autoref{sec:interview}.\newline

Essentially the motivation for developing an exciting and engaging edutainment game in this project is to make people interested in programming by providing a transition tool and having them consider getting an education in Software Engineering, so that the increasing demand on software in the future can be met by the number of educated software engineers.\citep{idaArtikelMangel}