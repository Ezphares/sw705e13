\section{Motivation}
\label{sec:motivation}

This section will explain the motivation for this project and will be concluded with a finalized problem statement.


Many people interact daily with computers without knowing how a computer works. There are many aspect of how a computer works, but one aspect is how programs are written. Software is included in an increasing number of products, and it is a field that keeps expanding. It is therefore important, that we get more people to realize, that computers are not too complicated and dangerous to work with. We might be able to get more people interested in learning programming languages, if we provide a transition tool, that teach people basic structures in programming, and how these structures can be combined to form working programs.


With the introduction of tablets and applications, the market for games has increased, and many educational games are available on these platforms; iOS, Android, and Windows mobile. It is important for us, that the product produced in this product can be used on many platforms and are therefore not restricted to ex. Apples App Store and Google Play. The motivation for creating a highly available product is to provide as many people with a transition tool, that makes basic programming skills more common.


It is important to state, that the motivation is not to teach users a specific programming language. Rather it is important for us, that constructs in \todo{maybe?} mostly imperative languages are understood, such as if-else-statements, loops, variable types and handling of variables, and that the user can get a visual representation of when a program is executed.


Essentially the motivation for developing a game in this project is to make people interested in programming by providing a transition tool and have them consider getting an education in Software Engineering, so that the increasing demand on software in the future can be meet by the number of educated software engineers.