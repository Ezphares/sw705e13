\section{Requirements}
\label{sec:requirements}

The requirements for this project are divided into practical (game-related) requirements and requirements with regards to the edutainment-oriented aspects of the project.
First the edutainment requirements gathered from the analysis will be presented, followed by the game-related requirements.

\subsection{Educational}

From the games examined in \autoref{sec:privateconsumers} and \autoref{sec:eduinsch} the following requirements have been established: 

\begin{itemize}
	\item The 'Global Conflict' and 'Where in the World Is Carmen Sandiego?' series focus on being helpful for different subjects such as social studies and medialogy.
While this is commendable, it is out of scope for this project.
This project should focus on teaching a single subject.
	\item The graphics of 'Global Conflicts' are mediocre compared to games put out by other commercial game studios.
	However, Global Conflicts is a popular game despite the lack of good graphics.
	This shows that there does not necessarily have to be a relation between the level of graphical detail in games, and the game's potential popularity.
	For this reason our game will be a 2D game without focus on high-budget graphics.
	\item 'The Oregon Trail' manages to make learning engaging without sacrificing the gameplay experience by depicting a realistic situation, that the player can interact with.
	The level of interaction should be included in our game as well.
	\item The browser-based games in general are very explicit in their intend to teach.
	It is a requirement, that our game avoids this issue, since our idea is to teach people basic programming constructs without explicitly stating "now you are learning!".
	\item The game must take care in how the level of abstraction is represented, because the game is intended to teach programming.
	The level of qualification the players have at the start is considered basic at best, therefore a higher level of abstraction is necessary to successfully teach.
	\item When the user interacts with the program, it is important that they write code that runs sequentially, such that the flow is predictable, as in Carnage Heart.
	This ensures that the program they write is easier to understand, but the program editor as well works in a predictable way.
\end{itemize}

\subsection{Practical}
\begin{figure}[ht]
\begin{center}
\begin{tabular}{|l|l|l|}
\hline
\multicolumn{3}{|c|}{Requirements}\\
\hline
Need & Want & Nice to Have\\
\hline
Graphics & Save/Load & Animation\\
Program Editor & Multiplayer & Leaderboards (Ranking system)\\
Environment & Challenges & Tutorial (Interactive)\\
Manual & AI & Export/Import\\
 & GUI/HUD & \\
\hline
\end{tabular}\newline
\caption{Need, Want, Nice to Have list of requirements}
\label{fig:req}
\end{center}
\end{figure}

The reasoning behind the priorities are as follows:

\begin{itemize}
	\item \textbf{Graphics} are needed, because the game emphasizes a visual approach to learning programming constructs and a console oriented approach would therefore not be viable.
	\item A \textbf{Program Editor} is needed to allow the users to interact with the game.
	\item \textbf{Environment} covers the world in which the user's code is executed.
	This is the 'playing field' of the game, where the users see the result of their programming and as such is essential to making the game engaging.
	\item A \textbf{Manual} is needed to teach users how to interact with the game.
	Optimally, the game should be designed to naturally teach the player how to play the game without a manual, but in the case that this is not achieved, a manual is essential.
\end{itemize}

\begin{itemize}
	\item \textbf{Save} and \textbf{Load} would be functions in the game, that would allow the user to save a program to the central database and load it into the editor and work on it further.
	Being able to save progress is a feature that we want to implement, since starting from scratch every time can be a source of frustration for the player, thereby making the game less enjoyable.
	\item \textbf{Multiplayer}, while not crucial, is an important aspect of the gamification of learning.
	It would allow users to compete against each other, which is a highly motivating factor.
	\item \textbf{Challenges} would allow the user to complete problems, that require a specific program construct while gaining a sense of accomplishment.
	\item \textbf{AI} is wanted as a replacement for multiplayer to have users compete against an opponent.
	Either AI or multiplayer should be implemented to make the game playable.
	\item \textbf{GUI/HUD} would be useful to display information about the entities in the game and let the user know how they are doing.
	This is different from Graphics, since Graphics only focus on visual representation of game, not every aspect of the world is included.
	This is included in the GUI/HUD 
requirement.
\end{itemize}

\begin{itemize}
	\item \textbf{Animation} on the entities is not crucial, however good animations serve to draw in visually inclined people and as the game is based on graphical programming, it would be a nice feature to have.
	\item \textbf{A leaderboard/ranking} system would be nice to promote the competitive element.
	As some people are fiercely competitive, having a leaderboard to climb could make them spend more time on the activity.
	\item An \textbf{interactive tutorial} would be nice to have for people who do not like to read game manuals, but want to get started right away.
	\item \textbf{Export/import} would allow the user to save their programs in files locally and work on them even while not connected to the internet.
	This would mean a lot to users who wish to spend time perfecting their program and would therefore be a nice feature to have, however it is not essential.
\end{itemize}