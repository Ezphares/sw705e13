\section{Requirements}\label{sec:requirements}
The requirements for this project are divided into practical or game-related requirements and requirements with regards to the edutainment-oriented aspects of the projects. First the edutainment requirements gathered from the analysis will be presented, then the practical requirements with regards to the game.

\subsection{Edutainment}
From the games the following requirements have been established: 
\begin{itemize}
\item The Global Conflict series focuses on being helpful for different subjects such as  social studies and medialogy. While this is commendable it is out of scope for this project. This project should focus on teaching a single subject very well. 

\item The graphics of the Global Conflict game are mediocre compared to games put out by commercial game studios and this may annoy some players because many people today are used to seeing excellent 3D-graphics in games. However, in this project, 2D-graphics could be utilized as these are not expected to be lifelike.

\item The tablet games are small, non-complex games focused on teaching single subjects, which should definitely be what this project should aspire to. However, these games are targeted at small children and as such considerations for a more mature audience should be made.

\item Oregon Trail manages to make learning engaging without sacrificing gameplay which is absolutely a requirement for this project as well.

\item The browser games in general are very explicit in their intent to teach something. This should be avoided in this project, seeing as the idea is to teach people something without explicitly stating that "now we are learning!". 
\end{itemize}

\subsection*{Practical}
\begin{tabular}{|l|l|l|}
\hline
\multicolumn{3}{|c|}{Requirements}\\
\hline
Need & Want & Nice\\
\hline
Graphics & Save/Load & Animation\\
Program Editor & Multiplayer & Leaderboards\\
Environment & Challenges & Tutrial (Interactive)\\
Manual & AI & Export/Import\\
 & GUI/HUD & \\
\hline
\end{tabular}

The reason for the priorities are as follows:
\begin{itemize}
\item Graphics are highly needed because the game emphasizes a visual approach to learning programming constructs.

\item A program editor is needed to allow the users to interact with the game.

\item Environment covers the world in which the user's code is executed. This is the "playing field" of the game, where the user sees the results of their programming and as such is highly need to make the game engaging.

\item A Manual is needed to teach users how to interact with the game.
\end{itemize}

\begin{itemize}
\item Save and Load are functions in the game that would allow the user to save a program to the central database and load it into the editor and work on it further.

\item Multiplayer, while not crucial, is an important aspect of the gamification of learning. It would allow users to compete against each other, which is a highly motivating factor.

\item Challenges would allow the user to complete problems that require a specific program construct while gaining a sense of accomplishment.

\item AI is needed as a replacement for multiplayer to have users compete against an opponent.

\item GUI/HUD would be useful to display information about the entities in the game and let the user know how they are doing.
\end{itemize}

\begin{itemize}
\item Animation. It is not crucial that the entities are animated, however good animations server to draw visually inclined people and as the game is based on graphical programming it would be a very nice feature to have.

\item Leaderboards would be nice to sharpen the competitive element. As some people are fiercely competitive, having a leaderboard to climb could make them spend more time on the activity.

\item Interactive tutorial would be nice to have for people who do not like to read game manuals but want to get started right away.

\item Export/import would allow the user to save their files locally and work on them even while not connected to the internet. This would mean a lot to users who wish to spend time perfecting their program.
\end{itemize}