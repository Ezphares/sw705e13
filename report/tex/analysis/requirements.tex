\section{Requirements}
\label{sec:requirements}

The requirements for this project are divided into practical (game-related) requirements and requirements with regards to the edutainment-oriented 
aspects of the project. First the edutainment requirements gathered from the analysis will be presented, followed by the game-related requirements.

\subsection{Edutainment}

From the games, the following requirements have been established: 

\begin{itemize}
	\item The Global Conflict series focuses on being helpful for different subjects such as social studies and medialogy. While this is commendable, it 
	is out of scope for this project. This project should focus on teaching a single subject well.

	\item The graphics of the Global Conflict game are mediocre compared to games put out by commercial game studios. This may annoy some players, 
	because many people today are use to seeing excellent 3D-graphics in games. However, in this project, 2D-graphics could be utilized as these are not 
	expected to be lifelike.\todo{Maybe we could talk somewhere about graphics vs. gameplay. Many players like games that are visually uninpressive, but 
	where the gameplay is fun. We could include Minecraft as an example or older Mario Games maybe?}

	\item Many tablet games are small, non-complex games focused on teaching single subjects, which should definitely be what this project should aspire 
	to. However, these games are targeted at younger children and as such considerations for a more mature audience should be made. It should be noted 
	that more commercial and complex games are finding their way onto tablets, e.x. Infinity Blade on the iOS platform.

	\item Oregon Trail manages to make learning engaging without sacrificing gameplay, which is absolutely a requirement for this project.

	\item The browser games in general are very explicit in their intent to teach something. This should be avoided, since the idea is to teach people 
	something without explicitly stating "now we are learning!". 
\end{itemize}

\subsection*{Practical}
\begin{tabular}{|l|l|l|}
\hline
\multicolumn{3}{|c|}{Requirements}\\
\hline
Need & Want & Nice\\
\hline
Graphics & Save/Load & Animation\\
Program Editor & Multiplayer & Leaderboards (Ranking system)\\
Environment & Challenges & Tutorial (Interactive)\\
Manual & AI & Export/Import\\
 & GUI/HUD & \\
\hline
\end{tabular}

The reasoning behind the priorities are as follows:
\begin{itemize}
\item Graphics are highly needed, because the game emphasizes a visual approach to learning programming constructs and a console oriented approach 
would therefore not be viable.

\item A Program Editor is needed to allow the users to interact with the game.

\item Environment covers the world in which the user's code is executed. This is the "playing field" of the game, where the user sees the results of 
their programming and as such is essential to make the game engaging.

\item A Manual is needed to teach users how to interact with the game. Optimally, the game should be designed to naturally teach the player how to play 
the game without a manual, but in the case that this is not achieved, a manual is essential.
\end{itemize}

\begin{itemize}
\item Save and Load are functions in the game, that would allow the user to save a program to the central database and load it into the editor and work 
on it further. Being able to save progress is a feature that we want to implement, since starting from scratch every time can be a source of 
frustration for the player, thereby making the game less enjoyable.

\item Multiplayer, while not crucial, is an important aspect of the gamification of learning. It would allow users to compete against each other, which 
is a highly motivating factor.

\item Challenges would allow the user to complete problems, that require a specific program construct while gaining a sense of accomplishment.

\item AI is needed as a replacement for multiplayer to have users compete against an opponent. Either AI or multiplayer should be implemented to make 
the game playable.

\item GUI/HUD would be useful to display information about the entities in the game and let the user know how they are doing. This is different from 
Graphics, since Graphics only focus on visual representation of game, not very aspect of the world is included. This is included in the GUI/HUD 
requirement.
\end{itemize}

\begin{itemize}
\item Animation. It is not crucial that the entities are animated, however good animations server to draw visually inclined people and as the game is 
based on graphical programming, it would be a nice feature to have.

\item Leaderboards/ranking system would be nice to sharpen the competitive element. As some people are fiercely competitive, having a leaderboard to 
climb could make them spend more time on the activity.

\item An interactive tutorial would be nice to have for people who do not like to read game manuals, but want to get started right away.

\item Export/import would allow the user to save their files locally and work on them even while not connected to the internet. This would mean a lot 
to users who wish to spend time perfecting their program and would therefore be a nice feature to have, however not essential.
\end{itemize}