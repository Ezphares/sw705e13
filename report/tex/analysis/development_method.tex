\section{Software Engineering Method}
\label{sec:software_engineering_method}

There are many different development methods, such as Scrum, \ac{xp}, \ac{up}, and the classic waterfall method.
This project relies heavily on initial analysis before starting to design and implement, since the problem was not clearly defined from the beginning.
Using \ac{xp} to quickly write code would therefore not be suitable for this project, and for that reason, it was quickly discarded.
Scrum is suitable to implement in projects where requirements for the final product often change.
It includes cycles, called sprints, that include analysis, design, implementation and testing.
We did not feel, that Scrum was suitable for our project, since the requirements, when determined, are fairly static.
The static nature of the requirements might make it possible to implement the classical waterfall method, but given our preferences from previous semesters, the waterfall method was not chosen.
Therefore, we decided to use \ac{up} as the development method.
\ac{up} divides the software development processes into the following four phases: 

\begin{itemize}
\item \textbf{Inception} The project's boundaries and scope should be defined, as well as risk identification and preliminary project schedule.
\item \textbf{Elaboration} The team is expected to create the majority of the system requirements and establish the system architecture.
\item \textbf{Construction} This should be the largest phase in the project, since this is where the system is implemented.
\item \textbf{Transition} The project is deployed and feedback is incorporated into future releases.
\end{itemize}

Each phase is divided into iterations and will contain most of the activities associated with software development (analysis, design, implementation and testing), but to varying degrees. As can be seen in\autoref{fig:up}, the main focus will gradually shift from one activity to the next in the process.

\begin{figure}[h]
  \centering
    \includegraphics[width=\textwidth]{img/Development-iterative.png}
  \caption{Phases and activities in UP \cite{UPfig}}
  \label{fig:up}
\end{figure}

The fact that \ac{up} is divided into phases that each have iterations fits well with the group's wish to focus mostly on one development activity at a time, without using the dated classical waterfall structure of the project.
\ac{up} also initially focuses on requirements and analysis, which is suitable for this project. However, \ac{up} is not specifically designed for web-development so some modifications are needed, such as including navigation architecture in the architecture design.\cite{UPcase}

The project theoretically has 15 weeks at its disposal, of these the first 2 should be spent on the inception phase, the next 4 on elaboration. Next, 6 weeks should be spent on construction and the last 3 weeks should be spent on transition. However, due to lectures and exercise sessions held during this time, these time frames should be viewed as absolute best case scenarios.