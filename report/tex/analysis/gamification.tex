\section{Gamification}\label{sec:gamification}
\todo{Maybe make it more clear what problem gamification attacks? Section explain gamification really good, but I miss some a little more analysis.}Gamification is "the process of game-thinking and game mechanics to engage users and solve problems" \cite{Zichermann2011}, i.e. using core elements of games to attract users and keep them coming back to a specific task.

Gamification is used in many areas, for example in engaging users in solving complex tasks: The game FoldIt, developed by University of Washington in 2008, employs people in folding proteins, which is a job that the brute-force approach of computers does poorly compared to man's natural abilities with regards to spatial reasoning and 3D pattern matching. In 2011 a team of users managed to decode an AIDS-causing monkey virus in just 10 days, a task which had been unsuccessfully attempted by scientists for 15 years.\cite{Huff2011} Today, the game has almost 500.000 registered users, using their spare time folding proteins for science for free.\cite{FoldIt2013}

FoldIt works, because it employs game mechanics and dynamics that correlate quite well with primary human desires to keep people interested and engaged in the activity.
Figure \ref{fig:bunchball} shows the interaction between human desires such as status, competition and rewards.

\begin{figure}[hptb]
  \centering
    \includegraphics[width=\textwidth]{img/bunchball.png}
  \caption{Interaction between Human desires and Game mechanics}
  \label{fig:bunchball}
\end{figure}

\todo{Really good analysis part of the section}As can be seen, the human desires can be fulfilled via various game mechanics. In FoldIt, for example, a Leaderboard allows the users to compete against each other either as groups or soloists, which allows both individualists and team players to participate with a feeling of fulfillment. 

If these game mechanics are used, many people will feel entertained while using the product. This is beneficial to a product that, for example, educates the user. In the online application FreeRice, developed for the World Food Program, the user's desire for altruism is fulfilled via donations of rice grains for completed tasks, while general academic skills are honed, such as vocabulary and basic mathematical proficiency. Again it is possible to join groups, and there is a leaderboard, thereby introducing an element of competition.\cite{freerice}

As has been described, the game mechanics in Figure \ref{fig:bunchball} will give people an extrinsic motivation (rewards, leaderboards etc) to keep playing, although the activity might not carry any intrinsic motivation for the user. This is useful in education, because people tend to easily give up, or simply never really start, when acquiring knowledge and skills that, while useful, may not be their primary field of interest. Keeping people entertained while learning can ensure that they spend more time on the subject and may also pique their interest in an area they might not have explored by themselves.