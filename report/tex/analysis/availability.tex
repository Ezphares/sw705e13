\section{Availability}
\label{sec:availability}
In computer science, availability is defined by the up time of a software product, meaning how often the product can be used.\cite{defAvailability}
In this report, we define good availability as the ability to easily obtain and use a piece of software.
It should also be noted, that we distinguish availability from accessibility, where accessibility is more focused on how easy a piece of software is to use for people with disabilities.
Accessibility is not a focus in this project and will not be discussed in further detail.\newline

The availability of games is a concern, when dealing with edutainment games.
Some games are only available in public schools, providing a license to a bundle of games, while others are sold commercially.
This limits the time a user can actively interact with the game and in the case of edutainment games, it will also limit the learning potential.
Commercial games have a higher degree of availability, since they are sold in stores which means a single license can be bought by anyone.\newline

With the widespread introduction of portable devices, such as tablets and smartphones, many games are now available on the go.
This serves a useful purpose, since children can interact with the games, for example on long trips or during other long periods of inactivity.
This also increases the availability of games.
Many game applications are miniature games, that are sold at a low price, around $\$0,99 (6,00 DKK)$.
The low price increases availability.
Also, given that many of the newest portable devices support fast mobile Internet (3G and 4G), it is possible for users to access the Internet everywhere.
Making a web based edutainment game is therefore mostly restricted by the availability of the Internet, be it either wifi or mobile.\newline

To achieve a high degree of availability, it is intuitive to make products for a widely used platform, the initial cost of entry should be low or directly reflect the quality of the product, and it has to have a high degree of visibility.
We define visibility as the ability of a product to be easily found by the customer.
Methods to increase visibility include, but are not limited to, advertisement and active use of social networks.