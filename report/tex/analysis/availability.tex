\section{Availability}
\label{sec:availability}

In computer science, availability is defined the up time of software product, meaning how often the product can be used. In this report, we define good availability as the ability to easily obtain and use a piece of software. It should also be noted, that we distinguish availability from accessibility, there accessibility is more focus on how easy a piece of software is to use for people with disabilities. The aspect of accessibility is not a focus in this project, and will not be discussed in further detail.\fixme{check: we do not talk more about accessibility} The usability of software concerns how easy it is to use a product. This aspect is important in any project and the results of our usability test can be seen in chapter \fixme{ref:to chapter about usability}.

The availability of games is a concern, when dealing with edutainment games. Some games are only available in public school, providing a license to a bundle of games, while others are sold commercially. This limits the time a user can actively interact with the game and in the case of edutainment games, it will also limit the learn potential. Other games, such as 'Magnus og Myggen', 'Pixiline', and 'Hugo', are created to a certain platform, where the focus of the game is having fun, and the educational aspect is secondary or hidden for the user in mini games within the game. An example could be learning about the solar system in order to make progress or order a subset of the basic elements from heaviest to lightest. These types of games have a high degree of availability, since they are sold in store and can be used on any computer.

With the widespread introduction of portable devices, such as tablets and smart phones, many games are now available on the go. This serves a useful purpose, since children can interact with the games, for example on long trips or during other long periods of inactivity. This also increase the availability of games. Many game applications are mini games, that are sold at a low price, around $\$0,99 (DKR. 6)$ . The low price increase availability. Also given, that many of the newer portable devices support mobile internet, it is possible to access the internet everywhere. Making a web based edutainment game is therefore only restricted by the availability to the internet, be it wifi or mobile.

To achieve a high degree of availability, it is important to use a platform that is widely used, the initial cost of entry should be low or directly reflect the quality of the product, and it has to have a high degree of visibility. We define visibility as the ability of a product of to be easily found by the customer. Methods to increase visibility include advertisement and active use of social networks.