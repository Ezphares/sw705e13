\section{Edutainment in Schools}
\label{sec:eduinsch}
In recent years edutainment in danish schools has increased in popularity, from being used in very seldom fashion, to a more structurized approach. 
This primarily observed by the arrival of various online internet portals for teachers to find edutainment oriented games, and the recent purchases of
iPads and other tablets or similar electronical devices for use during school. This change in opinion about edutainment resources is rather 
interesting, for that reason this section will describe what kinds of tools schools actually use, as part of their edutainment catalogue to teach and 
get an overview of what a good, useful game for a school is.


\subsection{Games in Schools}

The games currently used in schools are typically miniature, obscure titles, that are not well known or very famous outside of the schools. In other 
words, there are few edutainment titles, that has made a breakthrough in commercial terms as well. A few of the most commercially well known titles are 
Global Conflicts: Palestine, the Magnus og Myggen (Magnus and the Mosquito) series, and Pixeline. The most interesting of these titles, is Global 
Conflicts:Palestine and will be described in the following.


\subsection{What makes a game usefully edutaining?}

The criticism of videogames as part of the school's education catalogue, is primarily focused on the very definition of playing, which is that 'you 
play when the process is more important than the end result'. In other terms; education ceases and playing starts when the user of the tool does not 
necessarily care about what the outcome of their activity is. The main point by critics is then that entertainment is more closely associated to 
videogames, than videogames and entertainment. As described by\cite{edunoty}, the key to succesful edutainment is then to stop considering it both 
entertainment or education, as the semantics of those two principles, are that they are provided to you - you do not do anything to achieve these 
things, yourself. The change necessary is then to let the games used in schools, be more focused on the process of learning, as opposed to having an 
end result, to show for it, these are the key points to playful learning and in accordance to \cite{edunoty} the keys to a succesful 'edutainment' 
experience.

\subsubsection{Global Conflicts: Palestine}

The title was created by a small Danish company in tight collaboration with Copenhagen's IT University, and the game's purpose is to teach children 
about the highly complex conflict between Israel and Palestine. The title has done a phenomenal job in combining the act of learning and playing, and 
given the children who used it, valuable insight according to \cite{laeringpaaspil}. The game when bought with a school license is sold, in the 
package, with a task sheet for the children, to answer questions and solve puzzles about the game they have just played, making it a both 

real- and virtual-world experience for the children as they learn. Furthermore, it is possible for the teacher to monitor the children as they play 
through the game, via a separate computer and networking. Here the teacher can pin point when a child is having difficulties with a task, or sees an 
opportunity to further intensify a child's learning experience. In many ways Global Conflicts: Palestine is a game that has reshaped the opinion about 
playful learning in schools, and more importantly has kept the teacher in the center role in the childrens' playing experience, by ensuring that the 
child's playing experience is a focused one, and becomes a learning experience.\cite{laeringpaaspil}

\subsubsection{Tablet Games}

Most schools have already been using games which teaches the child about grammar, spelling or mathematics and so on, and some of these games have 
started emerging on a mobile platform as well. Typically sold as commercial games - i.e; not school licensed, they are accessible to everyone for a 
small fee - or in some situations, paid for via advertisements. An example of these kinds of games are, Trunky Learns Numbers or Trunky Learns Words, 
games created by Serious Games Interactive, who also created Global Conflicts: Palestine. These games are all very accessible to the children in Odder 
Kommune in Denmark, and are used to further enhance a child's learning experience\cite{odderipad}. It also suggests that edutainment games does not 
necessarily have to be large and complex as the previously mentioned game Global Conflicts: Palestine, something simpler is also valid. Unfortunately 
due to the very recent agreement to go through with these plans of bringing the iPads into the classrooms, it is too early to have enough valid data to 
document any positive change in a child's learning experience for doing this. However according to\cite{odderipadpjece}, the research and personnel 
group has noticed an increase in motivation, knowledge sharing, social competence and acceptance, as well as a focus on education that engages the 
students and a focus on delivering useful results amongst the students. The tablet however in this example, is used for more than playing various 
games, it is used to creatively engage the students, coordinate learning, and other IT oriented projects\cite{odderipadpjece}.