\section{Edutainment in Schools}
\label{sec:eduinsch}
The popularity of edutainment in danish schools have increased in recent years, from being seldom used to a more structured\todo{Right word?} approach. 
This can primarily be observed by the release of various web portals\footnote{Websites that serves as a gateway or a main entry point on the internet to a specific field-of-interest or an industry.\url{http://www.businessdictionary.com/definition/portal.html}} for teachers to find edutainment oriented games, and the recent purchases of iPads and other tablets or similar electronic devices for use both in schools and at home for homework.
This change in opinion about edutainment resources is rather interesting. For that reason, this section will describe what kind of tools schools use as part of their edutainment catalog to teach and provide an overview of what a good and useful game for a school is.


\subsection{Games in Schools}

The games currently used in schools are typically miniature, obscure titles, that are not well known outside of the school environment.
In other words, there are few edutainment titles, that have made a breakthrough in commercial terms.
A few of the most commercially well known titles are 'Global Conflicts: Palestine', the 'Magnus og Myggen' (Magnus and the Mosquito) and 'Pixeline' series.
We have chosen to focus on 'Global Conflicts: Palestine', because it is targeted at older children, which will be the primary focus group for our final product. The game will be described in the following subsection.

\subsection{Properties of Useful Edutainment}

The criticism of video games as part of the schools education catalog is primarily focused on the very definition of playing, which is that 'you play when the process is more important than the end result'.
In other terms, education ends and playing begins when the user of the tool does not necessarily care about what the outcome of their activity is and is more focused on the process of getting to a successful end.
The main point by critics is that entertainment is more closely associated with video games, than video games and education.
As described by \citep{edunoty}, the key to successful edutainment is to stop considering it as both entertainment and education, since the semantics of these two principles are that they are provided to the user.
The user does not have to do anything to achieve either entertainment and education.
Therefore, in order to make a good edutainment game, it should focus on the process of learning, not the end result, since the process is associated with gaming and when this process is combined with learning, edutainment is created.
These are the key points to playful learning and the keys to a successful 'edutainment' experience, according to \cite{edunoty}.

\subsubsection{Global Conflicts: Palestine}

Global Conflicts: Palestine was created by a small Danish company 'Serious Games Interactive' in close collaboration with Copenhagen's IT University.
The purpose of the game is to teach children about the highly complex conflict between Israel and Palestine.
In the game, the player is a reporter, who has to report on the war.
While playing, the player will create articles, include picture, and make headlines, that will shape the belief of the newspapers readers.
The game includes a basic communication system, where the player chooses between different dialog options to gain trust and uncover the true story behind the conflict. According to \cite{laeringpaaspil}:
\begin{quote}
	In general, Global Conflicts: Palestine proved to be a very successful and powerful alternative to the traditional primary use of books in Danish schools.
\end{quote}
The game, when bought with a school license, is sold in a bundle, that includes the game and a task sheet for the children.
The task sheet asks the children to answer questions and solve puzzles relevant to the game, right after playing.
Furthermore, it is possible for the teacher to monitor the children as they play through the game, via a separate computer using the same network as the children players.
This feature allows the teacher to pin point when a child is having difficulties with a task, or sees an opportunity to further increase the child's learning experience.\newline

In many ways 'Global Conflicts: Palestine' is a game, that has reshaped the opinion about playful learning in schools. It has been able to keep the teacher in the center role in the children' playing experience, by ensuring that the child's playing experience is focused, and becomes a learning experience.\cite{laeringpaaspil}\newline

Our product is not going to be designed specifically to be used in schools, so in that regard, 'Global Conflicts' is not relevant to the development of our game.
However, as stated earlier, the title is praised for the way that it combines learning and playing, so it is possible to learn from it.
The game is about a real conflict, which makes the game more engaging.
It forces the player to make important decisions, that can have consequences later in the game.
This aspect of including something real into our game to make it more engaging is indeed possible.\todo{We do not do this.}

Our game focuses on cells that attack other cells, much like white blood cell attack infection in the bloodstream. Giving the player a story, where he/she is part of the defense in a human host against invading infections could be a possible connection to make the game more real.\todo{Why do we not do this!?}

It may also be possible to include important decision making in the game, however, it should be noted, that 'Global Conflicts' is a third-person simulation game with characters that can interact with each other. This might not end up as part of our vision for our final product.

\subsubsection{Tablet Games}

Some schools already use games, that teaches the child about grammar, spelling, mathematics, etc., and some of these games have started emerging on a mobile platform as well.
Typically sold as commercial games (i.e; not school licensed), they are available to everyone for a small fee or in some situations mainly paid for by advertisers.
Two examples of these games are 'Trunky Learns Numbers' and 'Trunky Learns Words', which are games created by the company 'Serious Games Interactive'.
These games are available to the children in Odder Kommune in Denmark, and are used to further enhance a child's learning experience.\cite{odderipad}
The appearance of these games on the tablet market suggests that edutainment games do not necessarily have to be large and complex like 'Global Conflicts'.
Unfortunately, due the very recent agreement to go through with the plans of bringing iPads into schools, it is too early to have enough valid data to document any positive change in the learning experience of the children using tablets actively in their education.
However, the research and personnel group have noticed an increase in motivation, knowledge sharing, social competence and acceptance, as well as a focus on education that engages the students and a focus on delivering useful results amongst the students.\cite{odderipadpjece}
However, in this example the tablets are used for more than playing various games, but also used to creatively engage the students, coordinate learning, and other IT oriented projects.\cite{odderipadpjece}