\section{Edutainment in Schools}
\label{sec:eduinsch}
The popularity of edutainment in Danish schools has increased in recent years, from being seldom used to a more commonly used tool for learning. 
This can primarily be observed by the release of various web portals \footnote{Websites that serve as a gateway or a main entry point on the Internet to a specific field-of-interest or an industry.\url{http://www.businessdictionary.com/definition/portal.html}} for teachers to find edutainment oriented games and the recent purchases of iPads and other tablets or similar electronic devices for use both in schools and at home for homework.
This change in opinion about edutainment resources is rather interesting. For that reason, this section will describe what kind of tools schools use as part of their edutainment catalogue for teaching and provide an overview of what a good and useful game for a school is.


\subsection{Games in Schools}

The games currently used in schools are typically miniature, obscure titles, that are not well known outside of the school environment.
In other words, there are few edutainment titles, that have made a breakthrough in commercial terms.
A few of the most commercially well known titles are 'Global Conflicts: Palestine', the 'Magnus og Myggen' (Magnus and the Mosquito) and 'Pixeline' series.
We have chosen to focus on 'Global Conflicts: Palestine', because it is targeted at older children, which will be the primary focus group for our final product.

\subsection{Properties of Edutainment}

Edutainment is still debated as a term. Mitchel Resnick of MIT Media Laboratory dislikes it because we teach children from a young age that education is a passive activity, a service provided to you. And one that is boring and should be livened up with entertainment - another word he claims has passive, service-oriented connotations. Instead he prefers the term "playful learning", because playing and learning are activities that focus on the process instead of being means to an end. The key to successful playful learning, he says, is to use the child's intrinsic motivation and curiosity around a subject to help them learn about it. The challenge, he says, is that many people today view games and play as not having any value when it comes to teaching people anything.\cite{edunoty} 

In Mitchel Resnick's definition, what this project should achieve is playful learning, however it will continue to be called edutainment, due to the fact that his definition of playful learning is the same as the group's definition of edutainment. The properties of edutainment should be that it is

\begin{itemize}
\item Intrinsically motivated, i.e. the child should want to do the activity, not have an external source of motivation for it such as a teacher saying they must do it.
\item Focus on the creative process of trial and error without necessarily having any scientific approach or pre-existing hypothesis to test.
\item Give the child hands-on experience with the subject matter (e.g. mathematics or physics).
\end{itemize}

While the target audience in this project  may not be children, the properties listed above can also apply to older people, as curiosity and creativity are traits that do not disappear with age.

\subsubsection{Global Conflicts: Palestine}

Global Conflicts: Palestine was created by a small Danish company called 'Serious Games Interactive' in close collaboration with Copenhagen's IT University.
The purpose of the game is to teach children about the highly complex conflict between Israel and Palestine.
In the game, the player is a reporter, who has to report on the war.
While playing, the player will create articles, include pictures, and make headlines, that will shape the opinion of the newspaper's readers.
The game includes a basic communication system, where the player chooses between different dialogue options to gain trust and uncover the true story behind the conflict. 
\begin{quote}
	In general, Global Conflicts: Palestine proved to be a very successful and powerful alternative to the traditional primary use of books in Danish schools.\cite{laeringpaaspil}
\end{quote}
The game, when bought with a school license, is sold in a bundle, that includes the game and a task sheet for the students.
The task sheet asks the students to answer questions and solve puzzles relevant to the game, right after playing.
Furthermore, it is possible for the teacher to monitor the students as they play through the game, via a separate computer using the same network as the players.
This feature allows the teacher to pin point when a student is having difficulties with a task, or when an opportunity to further increase the student's learning experience arises.\newline

In many ways 'Global Conflicts: Palestine' is a game, that has reshaped the opinion about edutainment in schools. It has been able to keep the teacher in the center role in the students' playing experience, by ensuring that the student is focused, and playing the game becomes a learning experience.\cite{laeringpaaspil}\newline

The title is praised for the way that it combines learning and playing.
The game is about a real conflict, which makes it more engaging.
It forces the player to make important decisions, that can have consequences later in the game which also makes it engaging.

\subsubsection{Tablet Games}

Some schools already use games, that teaches the student about grammar, spelling, mathematics, etc., and some of these games have started emerging on mobile platforms as well.
Typically sold as commercial games (i.e not school licensed), they are available to everyone for a small fee or in some situations mainly paid for by advertisements.
Two examples of these games are 'Trunky Learns Numbers' and 'Trunky Learns Words', which are games created by the company 'Serious Games Interactive'.
These games are available to the children in Odder Kommune in Denmark, and are used to further enhance a student's learning experience.\cite{odderipad}
The appearance of these games on the tablet market suggests that edutainment games do not necessarily have to be large and complex like 'Global Conflicts'.
Unfortunately, due the very recent agreement to go through with the plans of bringing iPads into schools, it is too early to have enough valid data to document any positive change in the learning experience of the students using tablets actively in their education.
However, the research and personnel group have noticed an increase in motivation, knowledge sharing, social competence and acceptance, as well as a focus on education that engages the students and a focus on delivering useful results amongst the students.\cite{odderipadpjece}
In this example the tablets are used for more than playing various games. They are also used to creatively engage the students, coordinate learning, and do other IT oriented projects.\cite{odderipadpjece}