\section{Edutainment in Schools}
\label{sec:eduinsch}
In recent years edutainment in danish schools has increased in popularity, from being used in very seldom fashion to a more structured approach. 
This can primarily be observed by the arrival of various web portals\footnote{Websites that serves as a gateway or a main entry point on the internet to a specific field-of-interest or an industry.\url{http://www.businessdictionary.com/definition/portal.html}} for teachers to find edutainment oriented games, and the recent purchases of iPads and other tablets or similar electronic devices for use during school.
This change in opinion about edutainment resources is rather interesting. For that reason, this section will describe what kind of tools schools use as part of their edutainment catalog to teach and provide an overview of what a good, useful game for a school is.


\subsection{Games in Schools}

The games currently used in schools are typically miniature, obscure titles, that are not well known or very famous outside of the schools.
 In other words, there are few edutainment titles, that has made a breakthrough in commercial terms as well.
 A few of the most commercially well known titles are Global Conflicts: Palestine, the Magnus og Myggen (Magnus and the Mosquito) series, and Pixeline.
 We have chosen to focus on Global Conflicts:Palestine, because it is targeted to older children. The game will be described in the following paragraph.

\subsection{What makes a game usefully edutaining?}

The criticism of video games as part of the school's education catalog is primarily focused on the very definition of playing, which is that 'you play when the process is more important than the end result'.
In other terms; education ceases and playing starts when the user of the tool does not necessarily care about what the outcome of their activity is.
The main point by critics is then that entertainment is more closely associated to video games, than video games and education.
As described by \citep{edunoty}, the key to successful edutainment is then to stop considering it as both entertainment and education, since the semantics of those two principles, are that they are provided to the user. The user do not have to do anything to achieve these things, themselves.
Therefore, in order to make a good edutainment game, the game should focus on the process of learning, not the end result, since the process is associated with gaming and when the process is combined with learning we get edutainment.
These are the key points to playful learning and, according to \cite{edunoty}, the keys to a successful 'edutainment' experience.

\subsubsection{Global Conflicts: Palestine}

The title was created by a small Danish company in tight collaboration with Copenhagen's IT University. The purpose of the game is to teach children about the highly complex conflict between Israel and Palestine.
According to \cite{laeringpaaspil}, the title does a very great job at combining the act of learning and playing, and provide the children using the game with valuable insight into the conflict.
The game, when bought with a school license, is sold in a bundle that includes the game and a task sheet for the children.
The task sheet ask the children to answer questions and solve puzzles about the game, right after playing, thereby making it a both real and virtual world experience.
Furthermore, it is possible for the teacher to monitor the children as they play through the game, via a separate computer and networking.
This feature allow the teacher to pin point when a child is having difficulties with a task, or sees an opportunity to further intensify a child's learning experience.
In many ways Global Conflicts: Palestine is a game that has reshaped the opinion about playful learning in schools, and more importantly has kept the teacher in the center role in the children' playing experience, by ensuring that the child's playing experience is focused, and becomes a learning experience.\cite{laeringpaaspil}
\todo{Add:Describe what is the relation to our project if any.}

\subsubsection{Tablet Games}

Most schools have already been using games which teaches the child about grammar, spelling or mathematics and so on, and some of these games have started emerging on a mobile platform as well.
Typically sold as commercial games - i.e; not school licensed, they are accessible to everyone for a small fee - or in some situations, paid for via advertisements.
An example of these kinds of games are, Trunky Learns Numbers or Trunky Learns Words, games created by Serious Games Interactive, who also created Global Conflicts: Palestine.
These games are accessible to the children in Odder Kommune in Denmark, and are used to further enhance a child's learning experience.\cite{odderipad}
It also suggests that edutainment games does not necessarily have to be large and complex as the previously mentioned game Global Conflicts: Palestine.
Something simpler is also valid.
Unfortunately, due to the very recent agreement to go through with these plans of bringing the iPads into the classrooms, it is too early to have enough valid data to document any positive change in a child's learning experience for doing this.
However, the research and personnel group has noticed an increase in motivation, knowledge sharing, social competence and acceptance, as well as a focus on education that engages the students and a focus on delivering useful results amongst the students.\cite{odderipadpjece}
The tablet however in this example, is used for more than playing various games, it is used to creatively engage the students, coordinate learning, and other IT oriented projects.\cite{odderipadpjece}