\section{Alternative Game Testing}

As an alternative to employing an informal approach to testing - achieved by testing via interviews and questionnaires, and keeping it entirely open to a large online forum - would be to try a much more formal kind of testing. It would be interesting to see what the results of the test would be, if testing was done with this focus instead - would the same things or more be uncovered, or is the informal approach better? Regardless, with the current test results and kind of testing currently employed, we have successfully uncovered bugs and design problems with the game, that should be fixed if development on the project continues. A formal test could be conducted by making use of a 'standard' usability evaluation, wherein a test participant is brought into a usability laboratory, and they are tasked with completing a set number of tasks within the program. 


This kind of test would ensure that participants plays the game until they are done with their tasks, and as such they are forced to a much greater degree to learn, how the constructs function - instead of simply giving up and closing the browser, if they are so inclined. This would enable the test to be conducted in a more controlled environment, wherein a project group member was present in the laboratory with the tester, and help guide and steer the test, so as to avoid the participant ever being fully stuck. However this brings up a potential area of problem, that is not often addressed during this kind of user evaluation - namely the participants' stress level.


Consider a normal usability evaluation, here it is the software that is evaluated by a tester matching a demography. However, given that our game's objective is to teach, the participant may be subject to a lot of stress and frustration, when their knowledge will also be tested while on camera. After all, we seek to measure if a participant learns anything at all, from playing our game, and this could warrant undesirable side effects to a given test in the sense a participant may feel very much under the spotlight. Therefore this kind of User Evaluation may work entirely better, if the test's focus is shifted from being about how much they learn, to how easy they feel the game is to use from a Graphical Design perspective, to help ease any potential stress issues.