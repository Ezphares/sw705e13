\section{Problems with UP Implementation}
\label{sec:problems_with_UP}

The group chose to use the development method UP, because of the reasons described in \autoref{sec:software_engineering_method}. However, we were not very successful in utilizing the method during the development of our game.\\

We made frequent use of the iterative property of UP by having sprints from one meeting with our counsulor to the next. This worked well, and made sure that progress was always made either in the program or the report. Initially the focus was on the analysis part, which is also a part of UP, and we included design decisions early as well. However, the miss communication between group members lead to differences in what was written in the report and what was implemented. This miss communication meant, that some of the useful features from other games that we had examined in \autoref{sec:eduinsch} and \autoref{sec:privateconsumers} was never implemented. The UP method has the possibility to implement numerous different tools and techniques to help keep the project moving forward and in the right direction. For example, we never used diagrams to illustrate different aspects of the game, such as user interaction. A possible implementation of diagrams include the popular UML.\\

One of the larger underutilized aspect of UP in our project is the active use of testing. This include both testing of the code and for user experience. Testing of the code was done, but only in the late stages of development was unit testing actively implemented. Implementing unit testing early and making it a regular part of development would have been a better idea, where a test would be design for each function made right after it was done. This might seem as overkill, but the group would become more experienced using unit testing and the testing of the final product might therefore be easier and quickly done relative to a group of people with little experience with this form of testing the code. The testing of user and user interaction was not included earlier, because the product was far from completion. We only got the program working within the final two weeks of development, however a more active use of prototyping might have helped. The problem would then be to find people without much programming experience, that would be willing to use time to evaluate the prototypes.\\

UP is a good development method if followed properly, and we would advice the reader to actively study and understand the various tools suggested before choosing to use it as their development method in a project. The group did have a basic understanding of the method, but was unsuccessful in its implementation into the development. Increasing the amount of communication between group members and utilizing unit testing earlier would have helped the group, as well as implementing more of the tools, such as the UML diagram to increase the common understanding of the project, its goals and limitations.


