\section{Development Method}
\label{sec:problems_with_UP}

The group chose to use the development method UP, because of the reasons described in \autoref{sec:software_engineering_method}. However, we were not very successful in utilizing the method during the development of our game.\\

We made frequent use of the iterative property of UP by having sprints from one meeting with our supervisor to the next. This worked well, and made sure that progress was always made either in the program or the report. The progression of the project corresponded well with the illustration of the focus on different activities in \autoref{sec:software_engineering_method}. The UP method has the possibility to implement numerous different tools and techniques to help keep the project moving forward and in the right direction, but some of these were not included in the development process. For example, we never used diagrams to illustrate different aspects of the game, such as user interaction. A possible implementation of diagrams could include the Universal Modelling Language.\\

One of the larger underutilized aspect of UP in our project was the active use of testing. This include both testing of the code and for user experience. Testing of the code was done, but only in the late stages of development was unit testing actively implemented. Implementing unit testing early and making it a regular part of development would have been a better idea, where a test would be design for each function made right after it was done. The testing of user and user interaction was not included earlier, because the product was far from completion. We only got the program working within the final two weeks of development, however a more active use of prototyping might have helped. The problem would then be to find people without much programming experience, that would be willing to use time to evaluate the prototypes.\\

UP is a good development method if followed properly, and it could easily be modified to suit the development of a web-application. The group did have a basic understanding of the method, but was unsuccessful in its implementation into the development. Increasing the amount of communication between group members and utilizing unit testing earlier would have helped the group, as well as implementing more of the tools, such as the UML diagram to increase the common understanding of the project, its goals and limitations.

\subsection{Phases}
In this section we will reflect briefly on the different phases of development and discuss what went well, what could be improved and what experiences the group will utilize in future projects.

\subsubsection{Analysis}

The analysis phase of development went quite well, the analysis was thorough and well documented. Throughout the analysis the focus was very much on defining and scoping the problem to be solved and this proved quite challenging. However in the end the result were clear practical requirements along with well-founded requirements aimed at educating and entertaining the users. 

However as far as time is concerned it far exceeded the initial time frame set for analysis, which meant that other phases' times were cut short. This was in part due to poor planning on the group's part, but also due to a large amount of the lectures being held early in the semester, pushing this project into the background.

In future projects there should be more rigid deadlines early in the project to ensure that later phases do not become rushed. Additionally a concerted effort to define and scope the project quickly and early should be used.

\subsubsection{Design}

The initial design of the game was done fairly quickly but miscommunications in this phase meant documentation was lacking and some features were documented wrongly. Overall the final design reflected the main idea of providing an engaging way to learn about imperative programming constructs, and many planned features reflected the gamification elements from \autoref{sec:gamification}.

In the future the group will focus on documenting this phase better.

\subsubsection{Implementation}
The implementation phase was very rushed due to the delay early on in the project. This meant that there was little time to refactor code and not all features planned could be implemented. The resulting product was functional, testable and had all the crucial features, despite the group being pressed for time during this phase. In the future, the group will focus on starting the implementation on time to ensure that there is enough time to implement and re-implement to create a more polished end product.