\section{Development Method}
\label{sec:problems_with_UP}

The group chose to use the development method UP, because of the reasons described in \autoref{sec:software_engineering_method}. However, we were not very successful in utilizing the method during the development of our game.\\

We made frequent use of the iterative property of UP by having sprints from one meeting with our supervisor to the next. This worked well, and made sure that progress was always made either in the program or the report. The progression of the project corresponded well with the illustration of the focus on different activities in \autoref{sec:software_engineering_method}. The UP method has the option of implementing numerous different tools and techniques to help keep the project moving forward and in the right direction, but some of these were not included in the development process. For example, we never used diagrams to illustrate different aspects of the game, such as user interaction. A possible implementation of diagrams could include the Universal Modelling Language (UML).\\

One of the larger underutilized aspects of UP in our project was the active use of testing. This includes both testing of the code and for user experience. Testing of the code was done, but only in the late stages of development was unit testing actively implemented. Implementing unit testing early and making it a regular part of development would have been a better idea, where a test would be designed for each function right after it was done. The testing of user and user interaction was not included earlier, because the product was far from completion and therefore untestable. We only got the program working within the final two weeks of development, however a more active use of prototyping might have helped. The problem would then be to find people without much programming experience, that would be willing to use time to evaluate the prototypes.\\

UP is a good development method if followed properly, and it could easily be modified to suit the development of a web-application. The group did have a basic understanding of the method, but was unsuccessful in its implementation into the development. Increasing the amount of communication between group members and utilizing unit testing earlier would have helped the group, as well as implementing more of the tools, such as UML diagrams to increase the common understanding of the project, its goals and limitations.

\subsection{Phases}
In this section we will reflect briefly on the different phases of development and discuss what went well, what could be improved and what experiences the group will utilize in future projects.

\subsubsection{Analysis}

The analysis phase of development went quite well, the analysis was thorough and well documented. Throughout the analysis the focus was very much on defining and scoping the problem to be solved and this proved quite challenging. In the end the result was clear, practical requirements along with well-founded requirements aimed at educating and entertaining the users. 

However, as far as time is concerned it far exceeded the initial time frame set for analysis, which meant that other phases' times were cut short. This was in part due to poor planning on the group's part, but also due to a large amount of the lectures being held early in the semester, pushing this project into the background.

In future projects there should be more rigid deadlines early in the project to ensure that later phases do not become rushed. Additionally a concerted effort to define and scope the project quickly and early should be employed.

\subsubsection{Design}

The initial design of the game was done fairly quickly but miscommunications in this phase meant documentation was lacking and some design features were documented wrongly. Overall the final design reflected the main idea of providing an engaging way to learn about imperative programming constructs, and many planned features reflected the gamification elements from \autoref{sec:gamification}.

In the future the group will focus on documenting this phase better. Design planning meetings would be practical instead of allowing single project members to make design decisions on their own without consulting the entire group.

\subsubsection{Implementation}
The implementation phase was very rushed due to the delay early on in the project. This meant that there was little time to refactor code and not all features planned could be implemented. The resulting product was functional, testable and had all the basic features, despite the group being pressed for time during this phase. In the future, the group will focus on starting the implementation on time to ensure that there is enough time to implement and re-implement to create a more polished end product.

\subsubsection{Test}
With regards to functionality we found that Unit Test was a useful tool in testing our software in a very formal way. Though it was not all functions that posed errors, it was a chance as well to review some of the code, other members of the group had written, and as such understand their field of responsibility. So while the Unit Tests did not uncover errors, it helped share information on why some decisions in the code were made in an informal way. Furthermore it will help future development on the project, by formally stating what it takes to make a given function pass or fail a test of correctness.\newline

With regards to the online user test, the group was excited by the possibility of having the users of the entire Internet at their disposal, however such testing requires a big effort in spreading the link to the game and enticing people to answer the survey. This could have been done via a give-away after completing the survey where the users would receive a small gift. Given that it is also hard to make sure that people read the manual before playing, it would have been a huge boon to this test if the tutorials had been implemented allowing people to immediately start playing the game but also ensuring that they understood the interface and the rules. 
