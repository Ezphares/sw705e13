\section{Future Work}
\label{sec:future_work}
A natural extension of the project would be to take into consideration the answers from our survey and try to better the problems that users had with the program. Beyond that, we have a number of points on the requirements list that have not yet been fulfilled.

\subsection{User Feedback}
One of the things that users did not understand, was why they died on their first turn. This is most likely because they moved their cell out of the board, which currently kills the cell. There are multiple ways of correcting this misunderstanding. One of these ways could be to create a better message that the user would receive upon their death, it could tell the user that their cell died because it moved off the board. Another possible solution could be to not let the cells not spawn on the edge of the board.\newline

Another area where the users had problems understanding the game was the loop construct. It was especially unclear for most users how exactly the loop variable works, and how it is used in general. If possible this loop construct should either be reworked for easier understanding, or as an easier alternative we could create an example that the users could look at and hopefully get a better understanding of how it works. \newline


A way to help the players understand the program they have created, would be to create a debugging tool for the editor. This could include a minimap of the board on the Editor screen, which would constantly show a mini-version of the game. 

\subsection{Want and Nice to have from Requirements}
When looking at the neew, want and nice list from our requirements, there are a number of features that we have not been able to implement. It would be natural to start with the features that have been classified as "want", as these have the highest priority of the features not implemented. The three most obvious features that future work would begin with are tutorials, challenges, and multiplayer. A tutorial would allow the user to start playing the game immediately without reading a manual and challenges introduced in the tutorial would give us a chance to teach the user how a specific construct works. Multiplayer would make the game competitive which is very engaging for some users and allows for a social aspect of the game.

\subsection{Missing Mobile Support}
One of the main points about our game was to make it available on a large amount of platforms that have access to browsers that can run our game. However, it turns out, that the drag-and-drop function does not work on mobile devices. This is an issue, if we want the players to be able to play while on the go. The mobile devices did register button clicks. If we added the possibility in the editor to select a construct by clicking it and placing it on a tile by clicking that tile, the game would work on mobile devices with touch screen. Another possible solution would be to add the drag-and-drop functionality for mobile platforms.

\subsection{Gamification Mechanics}
One part of our problem statement was to implement gamification mechanics into our product in order to make our game more engaging and entertaining for older children. This was discussed in \autoref{sec:gamification}, however we did not implement any of the game mechanics. We did make the game ready to implement both challenges and multiplayer, where leaderboards would be a natural extension. If we in the future would utilize a leaderboard, it might make more sense to use a more detailed scoring system with whole positive numbers, that are calculated based on e.g. total energy of the winning team, time it took to win etc. Levels are also a natural possible extension of the game. This could be utilize through the challenges, where the player unlock access to more challenges when a set amount of challenges have been completed. It might also be possible to include individual player levels, meaning that a player would gain experience when winning over an opponent and in the future be matched with opponents of equal level if possible.\\

As can be seen, there are many different ways to implement some of the gamification mechanics in future work, and doing so might help the game gain popularity because of an increased amount of engagement, competition, sense of progression, and social interaction in multiplayer games.\newline

It is important to note that these mechanics wholly depend a game server being available to the users for prevention of cheating and validation of programs.

