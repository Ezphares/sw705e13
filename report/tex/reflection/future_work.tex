\section{Future Work}
A natural extension of the project would be to take into consideration the answers from our survey and try to better the problems that users had with the program. Beyond that, we have a number of points on the requirements list that we want or are nice, that have not been fulfilled.

\subsection{User Feedback}
One of the things that users did not understand, was why they died on their first turn. This is most likely because they moved their cell out of the board, which currently kills the cell. There are multiple ways of correcting this misunderstanding. One of these ways could be to create a better message that the user would receive upon their death, it could tell the user that their cell died because it ran off the board. Another possible solution could be to not kill cells when they run off the board, but instead teleport them to the other side of the board, this would eliminate the issue completely and possibly make it possible for more varied strategies.\newline

Another area where the users had problems understanding the game was the loop construct. It especially unclear for most users how exactly the loop variable works, and how it is used in general. If possible this loop construct should either be reworked for easier understanding, or as an easier alternative we could create an example that the users could look at and hopefully get a better understanding of how it works.


A way to help let players get the outcome they work for in the program more easily, would be to create a debugging tool as well for the editor. This could also include a sort of minimap overlay on the Editor screen, which would constantly show a mini-version of the environment (hexagonal grid) the programmer is creating the cell's program for. Possibly allowing the programmer to step-by-step execute / debug their program, and representing it on this minimap, could help the player envision how well their algorithm will perform in more powerful IDE's such as Visual Studio, and will address the problems some of the users encountered - finding it difficult to understand how the constructs worked.

\subsection{Want and Nice from Requirements}
When looking at the want and nice list from our requirements, there are a number of features that we have not been able to implement. It would be natural to start with the features that have been classified as want, as these have a higher priority than nice features. The three most obvious features that future work would begin with are AI, Challenges, and Multiplayer. These features would lift the game from an early prototype to a product that could actually be functional. Together with a good tutorial these features could teach new users how to use the game and hopefully learn some programming constructs. Features like AI, Challenges, and Multiplayer would also incentivise users to spend time playing the game, as there would be some goals to work towards, and reward the user for the time they invest in playing the game.

