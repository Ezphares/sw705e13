\section{Conclusion}
\label{sec:report_conclusion}

We approached the main problem in \autoref{sec:problemstatement} by making a game based on ideas from various different edutainment games both available in public schools and to the general public.\\

In comparison to the reviewed games this game incorporates a higher abstraction level than Code Combat, because users do not write actual code, instead they do graphical programming.

In contrast to Kodu Game Lab's preset commands this game allows for better customization of the programming constructs implemented.

Compared to Carnage Heart the instructions in our game correspond far more to general imperative control structures. \newline

In terms of requirements, all of the minimum requirements (\autoref{sec:requirements}) were fulfilled, however only a small subset of the other requirements were accomplished, see \autoref{sec:future_work}.

We did not manage to implement the gamification mechanics into our game, thereby possibly limiting the engagement and entertainment of the game. However, from the admittedly small user based evaluation of the game, we can conclude, that the basic idea behind the game is sound. Opening programming up to a large group of people using games is an idea, that could be worked more on in the future. It was described as a fun game by one user. The game was also described as not being too obvious about the learning process, which is a success for the project.\newline

All in all the group deems that it was successful in achieving a satisfactory solution to the problem statement even if the gamification mechanics are currently only in the design phase.