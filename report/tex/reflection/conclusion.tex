\newpage
\section{Conclusion}
\label{sec:report_conclusion}

The purpose of this project was to make a transition tool, that could possibly help deal with the increasing demand of engineers with programming experience.
The developed application should be a web based, and we chose to approach the problem from a gamification standpoint.
Through the analysis, we came to the problem statement:

\begin{quotation}
	How can an edutainment game be designed as a transition tool, to give people a taste of basic programming constructs, without making the learning process too obvious and scare them away before they know what it is.
	How can gamification techniques be utilized to make the game entertaining and engaging for older children to learn how to program or teach them constructs in the imperative paradigm?
\end{quotation}

We approached the main problem by making a game based on ideas from various different edutainment games both available in public schools and to the general public.
Minimum requirements for the project was established, and we will now conclude on the completion of the product in terms of the requirements set after the problem statement was define.\\

\textbf{Graphics} was defined as the game not being console based.
This requirement has been fulfilled.
The game is available on the web site \url{www.simonjensen.net}, and includes a menu to navigate, an editor, and the game itself.
The focus was not on making the game look good, so 2D graphics was implemented utilizing both webGL and the canvas element to draw the sprites.\\

A \textbf{Program Editor} was implemented, but not all constructs were implemented.
This will be described in further detail in the future work section, however, the editor is able to translate instructions into working JavaScript, that can be executed during the game.\\

The \textbf{Environment} was implemented as hexagon shaped board with hexagon tiles, based on our analysis.\\

A \textbf{Manual} was designed based on the implemented constructs.
The manual was user tested during the questionnaire, and the users found, that the manual was generally useful.\\

The basic requirements were fulfilled in the project, however, only the AI requirement from the wants requirements was also implemented.
Non of the other wants and nice requirements were implemented.
We added a health bar, which is a minimal part of the GUI/HUD requirement, but further work on the GUI/HUD would be necessary for us to deem the requirement fulfilled.\\

The final question is then whether the project was a success or failure.
In terms of requirements, all of the needs requirements were fulfilled, however only a very small subset of the other requirements were fulfilled.
We did not manage to implement any of the gamification mechanics into our game, thereby possibly limiting the engagement and entertainment of the game.
User testing the game showed \todo{insert reflection about test results}.