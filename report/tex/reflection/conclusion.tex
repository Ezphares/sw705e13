\section{Conclusion}
\label{sec:report_conclusion}

%The purpose of this project was to make a transition tool, that could possibly help deal with the increasing demand of engineers with programming experience. The developed application should be a web based, and we chose to approach the problem from a gamification standpoint. Through the analysis, we came to the problem statement:

%\begin{quotation}
%	How can an edutainment game be designed as a transition tool, to give people a taste of basic programming constructs, without making the learning process too obvious and scare them away before they know what it is.
%	How can gamification techniques be utilized to make the game entertaining and engaging for older children to learn how to program or teach them constructs in the imperative paradigm?
%\end{quotation}

We approached the main problem in \autoref{sec:problemstatement} by making a game based on ideas from various different edutainment games both available in public schools and to the general public.\\
%Minimum requirements for the project was established, and all of these have been fulfilled in the project.

%\textbf{Graphics} was defined as the game not being console based.
%This requirement has been fulfilled. The game is available on the web site \url{www.simonjensen.net}, and includes a menu to navigate, an editor, and the game itself. The focus was not on making the game look good, so 2D graphics was implemented utilizing both webGL and the canvas element to draw the sprites.\\

%A \textbf{Program Editor} was implemented, but not all constructs were implemented. This will be described in further detail in the future work section, however, the editor is able to translate instructions into working JavaScript, that can be executed during the game.\\

%The \textbf{Environment} was implemented as hexagon shaped board with hexagon tiles, based on our analysis.\\

%A \textbf{Manual} was designed based on the implemented constructs. The manual was user tested during the questionnaire, and the users found, that the manual was generally useful.\\


%The final question is then whether the project was a success or failure.
In terms of requirements, all of the minimum requirements (\autoref{sec:requirements}) were fulfilled, however only a small subset of the other requirements were fulfilled, see \autoref{sec:future_work}.
We did not manage to implement any of the gamification mechanics into our game, thereby possibly limiting the engagement and entertainment of the game. However, from the admittingly small user based evaluation of the game, we can conclude, that the basic idea behind the game is sound. That opening programming up to a large group of people using games is an idea, that could be worked more on in the future. It was described as a fun game by one user, but the same user did not find, that the game managed to teach anything new, so the product succeeded somewhat in the engaging requirement of the product, but it still lacks behind in the educational aspects. The game was also described as not been too obvious about the learning process, which is a success for the project. Overall we must conclude, that we need more test results to make a final conclusion.